%
% Copyright 2014, NICTA
%
% This software may be distributed and modified according to the terms of
% the BSD 2-Clause license. Note that NO WARRANTY is provided.
% See "LICENSE_BSD2.txt" for details.
%
% @TAG(NICTA_BSD)
%

\subsection{Empty frame allocation}

\begin{figure}[htb]
\begin{center}
\begin{msc}
msc {
    hscale = "1.5";

    R [ label ="Root Server"],
    C [ label = "Client"];
    
    C => R [ label = "rootserver\_clientid\_cap.mk\_map\_window(vaddr, size)"];
    R >> C [ label = "Reply(SUCCESS, win\_cap)"];
    C => R [ label = "rootserver\_clientid\_cap.file\_map(\"zero\_fill\", win\_cap)"];
    R box R [ label = "validate request and allocate win\_id"];
    R >> C [ label = "Reply(SUCCESS)"];
    C box C [ label = "touch anywhere in window"];
    C => R [ label = "rootserver\_clientid\_cap.pagefault(vaddr)"];
    R box R [ label = "transform into win\_id, offset"];
    R box R [ label = "found the mem object zero\_fill in itself"];
    R box R [ label = "allocate new frame and back the content by zero\_fill"];
    R :> C [ label = "rootserver\_clientid\_Internal.mapframe(frame)"];
    R >> C [ label = "Reply(SUCCESS)"];
}
\end{msc}
\end{center}
\caption{Using zero fill to allocate an empty frame.}
\end{figure}

\subsection{Share a memory object}

\begin{figure}[htb]
\begin{center}
\begin{msc}
msc {
    hscale = "1.5";

    R [ label ="Root Server"],
    P [ label = "Pager"],
    A [ label = "Client A"],
    B [ label = "Client B"];
    
    A box A [ label = "opened some object backed by pager in window A with win\_capA"];
    --- [ label = "assume A has the cap to talk to the pager and they can talk to each other"];
    A => P [ label = "pager\_clientAid\_cap.mk\_mobject\_anon\_cap(win\_capA, perm)"];
    P box P [ label = "validate request"];
    P >> A [ label = "Reply(SUCCESS, mobject\_anon\_cap)"];
    A => B [ label = "clientB\_clientAid\_cap.share\_object(mobject\_anon\_cap, win\_size)"];
    B >> A [ label = "Reply(SUCCESS)"];
    B box B [ label = "create a new window indicated by win\_capB"];
    B => P [ label = "mobject\_anon\_cap.object\_share\_init(win\_capB)"];
    P box P [ label = "validate request and find win\_capA"];
    P => R [ label = "rootserver\_pagerid\_cap.map\_share\_copy(win\_capA, win\_capB)"];
    R box R [ label = "find frames win\_capA\_frames mapped to win\_capA and book keep"];
    R box R [ label = "set up pager information for win\_capB"];
    R :> B [ label = "rootserver\_clientBid\_Internal.mapframe(win\_capA\_frames)"];
    R >> P [ label = "Reply(SUCCESS, win\_idB)"];
    P >> B [ label = "Reply(SUCCESS)"];
}
\end{msc}
\end{center}
\caption{Protocol for sharing a memory object from client A to client B.}
\end{figure}


\subsection{Shared buffer initialisation in A for argument passing to B}

\begin{figure}[htb]
\begin{center}
\begin{msc}
msc {
    hscale = "1.5";

    A [ label = "Client A"],
    B [ label = "Client B"];
    
    A box A [ label = "create memory object buffer X"];
    A box A [ label = "create memory object for start and end index/pointer"];
    |||;
    A box B [ label = "share X - read only for B"];
    A box B [ label = "share end index/pointer - read only for B"];
    A box B [ label = "share start index/pointer - R/W for B"];
}
\end{msc}
\end{center}
\caption{TX buffer initialisation.}
\end{figure}

\subsection{Shared buffer initialisation in A for receiving information from B}

\begin{figure}[htb]
\begin{center}
\begin{msc}
msc {
    hscale = "1.5";

    A [ label = "Client A"],
    B [ label = "Client B"];
    
    A box A [ label = "create memory object buffer X"];
    A box A [ label = "create memory object for start and end index/pointer"];
    |||;
    A box B [ label = "share X - R/W for B"];
    A box B [ label = "share end index/pointer - R/W for B"];
    A box B [ label = "share start index/pointer - R/W for B"];
}
\end{msc}
\end{center}
\caption{RX buffer initialisation.}
\end{figure}

\clearpage

Do we need a name server for this?

\subsection{Name registration and query}

\begin{figure}[htb]
\begin{center}
\begin{msc}
msc {
    hscale = "1.5";

    N [ label ="Name server"],
    S [ label = "Server"];
    
    --- [ label = "registration"];
    S box S [ label = "create a new memory object with mobject\_cap"];
    S => N [ label = "nameserver\_serverid\_cap.object\_name\_reg(mobject\_cap, \"mobject\_name\")"];
    N box N [ label = "internal book keeping"];
    N >> S [ label = "Reply(SUCCESS)"];
    |||;
    --- [ label = "query"];
    S => N [ label = "nameserver\_serverid\_cap.object\_name\_query(\"mobject\_name\")"];
    N box N [ label = "find the object according to \"mobject\_name\""];
    N >> S [ label = "Reply(SUCCESS, mobject\_cap)"];
}
\end{msc}
\end{center}
\caption{Object name registration and query.}
\end{figure}

\begin{figure}[htb]
\begin{center}
\begin{msc}
msc {
    hscale = "1.5";

    N [ label ="Name server"],
    R [ label = "Root server"];
    
    --- [ label = "registration"];
    R box R [ label = "receive call for server registration with name \"server\_name\""];
    R -> N [ label = "nameserver\_AsyncEP.server\_name\_reg(\"server\_name\")"];
    N box N [ label = "internal book keeping and generate server\_id"];
    N => R [ label = "rootserver\_nameserverid\_cap.name\_reg\_ack(SUCCESS, server\_id)"];
    R box R [ label = "book keep server\_id with server cap"];
    R >> N [ label = "Reply(SUCCESS)"];
    |||;
    --- [ label = "query"];
    R box R [ label = "receive call to find server of name \"server\_name\""];
    R -> N [ label = "nameserver\_AsyncEP.server\_find(\"server\_name\")"];
    N box N [ label = "find \"server\_name\" with its server\_id"];
    N => R [ label = "rootserver\_nameserverid\_cap.name\_query\_ack(SUCCESS, server\_id)"];
    R >> N [ label = "Reply(SUCCESS)"];
}
\end{msc}
\end{center}
\caption{Server name registration and query.}
\end{figure}
\begin{figure}[htb]
  \begin{center}
    \begin{msc}
      msc {
      	hscale = "1.75";
      	
        R [ label ="Root Server"],
        P [ label = "Pager"],
        C [ label = "Client"];

        C box C [ label = "touch a page"];
        C=>R [label = "rootserver\_clientid\_cap.pagefault(vaddr)"];
        R box R [ label = "page directory access"];
        --- [ label = "if need swap I/O, otherwise go to NO\_SWAP"];
        R box R [ label = "check if any vaddr in any window should sit in page at this PDE"];
        --- [ label = "YES, otherwise go to NO"];
        R box R [ label = "alloc swap buffer(might need swapping) and write/map victim page"];
        R->P [ label = "pager\_AsyncEP.notifyfault(win id, offset, op, swap\_vaddr)"];
        P box P [ label = "write page to storage or find/store page in pager address space"];
        --- [ label = "if page fault"];
        P box P [ label = "repeat fault process for this pager"];
        ...;
        --- [ label = "otherwise"];
        P=>R [ label = "rootserver\_pagerid\_cap.map(win\_id, page\_num, offset, perms)"];
        R box R [ label = "validate message and lookup pager page table for frame"];
        R>>P [ label = "Reply(SUCCESS)"];
        R box R [ label = "GO TO STAGE 2"];
        --- [ label = "NO"];
        R box R [ label = "handle paging internally"];
        ...;
        --- [ label = "NO\_SWAP"];
        R box R [ label = "page table alloc if new"];
        --- [ label = "STAGE 2"];
        R box R [ label = "page table access"];
        --- [ label = "if need swap I/O, otherwise go to STAGE MAP"];
        R box R [ label = "check if fault vaddr is in any window"];
        --- [ label = "YES, otherwise go to NO"];
        R box R [ label = "alloc swap buffer(might need swapping) and write/map victim page"];
        R->P [ label = "pager\_AsyncEP.notifyfault(win\_id,offset,op)"];
        P box P [ label = "write page to storage or find/store page in pager address space"];
        --- [ label = "if page fault"];
        P box P [ label = "repeat fault process for this pager"];
        ...;
        --- [ label = "otherwise"];
        P=>R [ label = "rootserver\_pagerid\_cap.map(win\_id, page\_num, offset, perms)"];
        R box R [ label = "validate message and lookup pager page table for frame"];
        R>>P [ label = "Reply(SUCCESS)"];
        R box R [ label = "GO TO STAGE MAP"];
        --- [ label = "NO"];
        R box R [ label = "handle paging internally"];
        ...;
        --- [ label = "STAGE MAP"];
        R box R [ label = "frame alloc if new"];
        R:>C [ label = "rootserver\_clientid\_Internal.mapframe(frame)"];
        R>>C [ label = "Reply(SUCCESS)"];
      }
    \end{msc}
  \end{center}
  \caption{More detailed version of page fault.}
\end{figure}

\begin{description}
  \item
  \begin{description}
  \item[\ty{mk\_mobject\_anon\_cap(win\_cap, perm)}] A method
    requesting the pager to create an anonymous memory object
    capability that is already mapped to a specific window. This
    anonymous capability then can be passed to other process for
    setting up sharing of the same copy mapped to the window indicated
    by \ty{win\_cap}. \ty{perm} indicates the permission that the
    caller client wants other processes to have when they gain access
    to the copy.
  \end{description}

\item[\ty{clientB\_clientid\_cap}] A distinguished badged client B cap
  to identify a client.
  \begin{description}
  \item[\ty{share\_object(mobject\_anon\_cap, win\_size)}] A call to
    client B by another client to share the memory object anonymous
    capability \ty{mobject\_anon\_cap} so that client B can make
    further sharing request to pager and indirectly root
    server. \ty{win\_size} indicates the suggested size of the window
    that can be used to accommodate the memory object mapping.
  \end{description}

\item[\ty{mobject\_anon\_cap}] An anonymous badged cap used to start a
  memory object sharing process. The server which owns the memory
  object listens to this capability.
  \begin{description}
  \item[\ty{object\_share\_init(win\_cap)}] A call to the server that
    listens to the anonymous capability. It asks the server to set up
    pager information for this argument \ty{win\_cap} and indicates
    that the caller wants to gain access to the copy of this object
    that is mapped to another window. The server will then find the
    window which has the object mapped and pass it to the root server
    along with the argument \ty{win\_cap} for further actions.
  \end{description}

\item[\ty{rootserver\_pagerid\_cap}] A distinguished badged root
  server cap to identify a pager.
  \begin{description}
  \item[\ty{map\_share\_copy(win\_capA, win\_capB)}] A call to the
    root server from a pager for mapping the mapped frame to another
    process again in order to create the shared state. Root server
    finds the frames mapped to the window indicated by ty{win\_capA}
    and maps them again to the window indicated by ty{win\_capB}. It
    also performs pager registration of the caller pager to the window
    indicated by \ty{win\_capB}.
  \end{description}

\item[\ty{nameserver\_serverid\_cap}] A distinguished badged name
  server cap to identify a server.
  \begin{description}
  \item[\ty{object\_name\_reg(mobject\_cap, mobject\_name)}] A call to
    the nameserver for registering a name for a memory object.
  \item[\ty{object\_name\_query(mobject\_name)}] A call to the
    nameserver for finding the name of a memory object. Name server
    should find the name and return a the ty{mobject\_cap} to the
    server.
  \end{description}


\item[\ty{nameserver\_AsyncEP}] An async endpoint used by the
  nameserver to receive request from root server.
  \begin{description}
  \item[\ty{server\_name\_reg(server\_name)}] An asynchronous call to
    the nameserver for registering a name for a server. Name server
    should register the name and return a server\_id.
  \item[\ty{server\_find(server\_name)}] An asynchronous call to the
    nameserver for finding the name of a server. Name server should
    find the name and return a server\_id.
  \end{description}
  
\item[\ty{rootserver\_nameserverid\_cap}] A distinguished badged root
  server cap to identify a name server.
  \begin{description}
  \item[\ty{name\_reg\_ack(RETURNCODE, server\_id)}] A call to the
    root server for acknowledging the asynchronous registration
    request with a return code and \ty{server\_id}.
  \item[\ty{name\_query\_ack(RETURNCODE, server\_id)}] A call to the
    root server for acknowledging the asynchronous name query request
    with a return code and \ty{server\_id}.
  \end{description}
\end{description}



%%% Local Variables: 
%%% mode: latex
%%% TeX-master: "paper"
%%% End: 
