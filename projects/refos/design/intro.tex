%
% Copyright 2014, NICTA
%
% This software may be distributed and modified according to the terms of
% the BSD 2-Clause license. Note that NO WARRANTY is provided.
% See "LICENSE_BSD2.txt" for details.
%
% @TAG(NICTA_BSD)
%

This document provides the abstract protocols for a multi-server microkernel based operating system.

The overall design revolves around the abstraction of a dataspace. A dataspace is a memory space, a
series of bytes, representing anything from physical RAM to file contents on disk to a device or
even to a random number generator. The concept is analogous to UNIX files, which may represent
anything from \texttt{/usr/bin/sh} to \texttt{/dev/audio} to \texttt{/dev/urandom}.

This dataspace abstraction which dataspace servers provide paves the way for more complicated
infrastructure on top such as sharing with complex trust relationships, memory management, file
systems, and distributed naming. On top of this additional infrastructure, the OS may implement more
features such as POSIX standard syscalls, ports of existing software, drivers, display servers, and
so forth.

% ----------------------------------------------------------------------
\section{Terms}

This section will attempt to briefly clarify term ambiguities before the details
are presented. The brief descriptions also quickly give an overview of the
terms.

\texttt{Methods} are conceptual "function calls", although they are usually
implemented via communication with another process, in which case they are
actually remote procedure calls. Regardless of the underlying implementation,
methods here refer to the actual procedures of an interface which get invoked.

\texttt{Interface} is a collection of (usually related) methods, that
usually serve to abstract a conceptual object. For instance, the \texttt{c}
standard library functions \texttt{fopen}, \texttt{fscanf}, \texttt{fprintf},
\texttt{fclose} ..etc form an abstract interface and provide abstraction for a
\emph{file} object.

\texttt{Message Sequence} is an ordered sequence of method invocations,
replies, event notifications or operations which may achieve some overall
result or goal.  Example message sequences using the described protocols for
commonly needed goals will be presented in this document.

\texttt{Protocol} here will refer to a collection of interfaces, their methods,
message sequences and system components.

\texttt{Dataspaces} are abstractions over memory objects, analogous to UNIX files. A dataspace
may be opened / closed, written to or read from, mapped into client memory, and may represent
a device, a file, or simply anonymous memory.

\texttt{Servers} are normal processes which have registered themselves using
the naming service so that other client processes may find. Once the client
finds the process, the client may then connect and establish a session with the
server, and the server is given access to the notification of client deaths in
order to do any client book-keeping it may require.

\texttt{Dataspace servers} (henceforth also referred to as \emph{dataservers}) are
servers which implement the dataspace interface (or a subset of it), in order to
provide a dataspace service to clients. Dataspace servers may be most likely
designed to provide the dataspaces which represent a common concept. For
instance, a file server may implement the dataspace interface for files stored
on disk, while an audio driver may serve audio bytestream dataspaces.

\texttt{Process servers} are servers which implement the process server interface,
and provide process abstraction. In practice, the process server may also need
to implement additional interfaces and provide additional functionality in order
for a system to start. For instance in our sample implementation the process server
also does naming, console input/output, and implements the dataspace interface
for anonymous memory and acts as the memory manager.

\texttt{Pagers} are servers which implement the pager service interface, which
is very closely related to the dataspace interface, and can be considered an
optional feature of a dataspace server. Dataspace servers which act as pagers
may have its dataspaces be mapped directly into client virtual memory.

% ----------------------------------------------------------------------
\section{Microkernel}

The \refOS protocol is designed with an \Lf family microkernel in mind, and
assumes the existence of features such as inter-thread IPC and capabilities.
This section describes in detail the notable assumed features, with brief
descriptions of such features.  These features are available in advanced
\Lf-based microkernels such as \seLf.

\subsection{IPC Messages}

\begin{itemize}

	\item 
	\textbf{Synchronous Calls} are used to invoke methods implemented by a
	server. They involve a badged cap invocation (of a synchronous endpoint or
	SEP) that identifies the server and represents the authority of the caller.
	The caller blocks until the server responds. \emph{Note: A caller
	implicitly trusts the server it calls.} Calling an untrusted server is
	possible, but should be avoided in general -- which we strive to do in
	\refOS.

	Exceptions, such as page faults, are propagated via sending and blocking on
	a SEP.

	\item 
	\textbf{Reply} via reply caps. A call via invoking a SEP generates a Reply
	capability for the server to respond to the caller. Thus the callee can
	send a response to the original caller, and unblock the caller, by replying
	with the reply capability. Note that reply caps are guaranteed to either
	succeed or fail and not block. This is required for a trusted server to
	safely reply to an untrusted client.

	\item
	\textbf{Asynchonous Notifications} are sent via a \texttt{Send} to an
	asynchronous end point (AEP) capability. Notifications are non-blocking
	with at-least-once semantics (the bits are OR'd together). This AEP can be
	used to notify an untrusted client. To send messages(as opposed to just
	notifications) to an untrusted client, shared memory must be set up between
	the two. Care must be taken to avoid blocking, denial of service, etc... on
	and shared memory buffer which may be filled up.

\end{itemize}

\subsection{Capability Access Model}

\refOS assumes the microkernel provides support for a capability-based access
model, and more importantly, the ability to transfer badged capabilities via
IPC. 

The basic concept of a capability (henceforth referred to as \emph{cap}) is that
an object representing the permission to access a particular object. This is
similar in idea to UNIX file descriptors, which represent the access that the
process has to the file. Just like a UNIX file descriptor table, the capability
objects are to be kept in kernel memory. User programs only have handles to
such objects, and cannot directly manipulate them.

The \refOS is designed with the assumption that synchronous call and reply IPC
messages may contain capabilities, and that there is some way to store a
minimal amount of immutable information in such capabilities, and that the
receiving process may read the immutable information from the capabilities that
it has received via IPC. 

In the sample implementation given using the seL4 microkernel, this is
implemented as kernel IPC endpoint \emph{badges}; endpoint capabilities may be
badged with an integer, and the badge of an endpoint once it has been badged is
immutable. When a badged endpoint cap is sent along the same endpoint the one
it is pointing to, then the kernel automatically \emph{unwraps} the badge to be
read. Therefore, only the process which created the badge can read it back via
IPC. \refOS makes heavily use of these badges in order for servers to track
server objects (such as windows and dataspaces).

\subsection{Virtual Memory Support}

\refOS assumes that the underlying kernel exposes virtual memory management.
This is a standard feature of microkernels. It is also assumed that the kernel
is able to deliver page fault exceptions via IPC to a memory manager thread,
and be able to resolve the fault at a later stage by replying to the IPC.

% -----------------------------------------------------------------------------
\section{Notation}

This section outlines the notation used to describe the \refOS protocol.
Specifically, it looks at the messaging between components that is used to
develop higher-level protocols. 

\subsection{Operation Types} 

\autoref{fig:mta} presents the four basic operations we are using (and
indirectly attempting to limit ourselves to). They are synchronous calls (i.e.
a method invocation on an object), replies, asynchronous notifications, and
kernel system calls that asynchronously modify the state of another process
(such as mapping memory directly into the virtual address space of another
process).

The synchronous / asynchronous calls and reply operations may be implemented
directly using the microkernel features described above, such as IPC,
capabilities, the ability to send capabilities via IPC, virtual memory
operations, and so on.

\begin{figure}[htb]
  \centering
  \begin{msc}
    msc {
      A, B;
      A => B [ label = "A synchronous call or fault via a SEP"];
      A << B [ label = "A return from a call (via reply cap)"];
      A -> B [ label = "An asynchronous signal via an AEP"];
      A :> B [ label = "A kernel operation that affects B"];
    }
  \end{msc}
  \caption{Message Type Arrows.}
  \label{fig:mta}
\end{figure}

\subsection{Message Notation} 

Protocols are shown as series of messages. Messages are implemented by operations,
and the types of different messages directly correlate to the operation types.
This section outlines the notation which this document will henceforth adhere to
when describing the various forms of messages that form the \refOS protocol.

\subsubsection*{Method Invocation}

Most of the interface methods are synchronous. The client making the call is
considered to be less trusted than the server that is receiving the call. As
the client is making the call, it is blocked until the server finished handling
the call and replies via the reply operation. When the server replies, the call
is said to be finished and the client resumes execution. Method invocation is
implemented with the synchronous call operation via a SEP. Capabilities may be
passed via method invocation, and also via the corresponding reply.

For synchronous method invocations by a less trusted client to a more trusted
server, the following notation is used:

\begin{center}
  \pro{server}{interface}{method(arg1, arg2, ...)}{(val1, val2, val3 ...)}
\end{center}

Where:

\begin{itemize} 
	\item
	\srv{server} indicates the name of the server that is receiving and
	handling the method invocation via an endpoint. Eg. \srv{procserv} and
	\srv{dataserv} will henceforth be used in method descriptions to referr to
	a process server and a data server respectively.
	
	\item
	\obj{interface} refers to the name of the interface that the server
	implements.

	\item
	\texttt{method(arg1, arg2, ...)} refers to the name of the interface
	method, \texttt{(arg1, arg2, ...)} representing the arguments of the method
	call.

	\item
	\texttt{(val1, val2, val3 ...)} representing the set of return values, output variables
	and/or reply capabilities of the method invocation.
\end{itemize}

For example, the notation \met{dataserv}{dataspace}{open} means an invocation
of the \texttt{open} method of the \obj{dataspace} interface on the server
called \srv{dataserv}. Note that any \srv{server} can implement a particular
\obj{interface}.  The encoding of method and arguments in the actual message is
left as an implementation detail.

\autoref{fig:epm} describes an example protocol description (in the form of a
message sequence diagram) of a client process invoking a method via an
synchronous message operation:

\begin{figure}[htb]
  \centering
  \begin{msc}
    msc {
      R [ label ="Server"],
      C [ label = "Client"];
      C => R [ label = "server\_process\_C.method(13, id)"];
    }
  \end{msc}
  \caption{Example protocol message sequence diagram.}
  \label{fig:epm}
\end{figure}

\subsubsection*{Event Notification}   

Some servers need a method of notifying another process without blocking and
waiting for the other process to finish processing it. Asynchronous signal
operations via asynchronous endpoints are used to send such notifications to
other servers. Capabilities may not be transferred via a notification.

Note that \refOS does not make the assumption that the kernel is able to pass
message contents via the notification itself, just that the kernel is able
to pass just the notification. In order for the actual message to be delivered
along with the notification, the implementation of the protocol may for instance
make use of a shared buffer.

The following notation is used for event notifications:

\begin{center}
\met{server}{event}{notify(arg1, ...)}
\end{center}

This notation is purposely very similar to a method invocation, as those
two are similar in concept in that they are both inter-process messages.
The different message sequence diagram arrows used for synch calls
and async notifications clearly distinguish between the two.

\subsubsection*{Abstracted protocols}

Sometimes the protocols are of different level of abstractions. In order to
separate two closely related protocols in one's sequence diagram, we use the
following representation as indicated in \autoref{fig:separate_p}:

\begin{figure}[htb]
  \centering
  \begin{msc}
    msc {
      A [ label ="Process A"],
      B [ label = "Process B"];
      A abox B [ label = "A separate protocol goes here."];
      A => B [ label = "processB\_object\_C.method(arg, ...)"];
    }
  \end{msc}
  \caption{Example of separation of protocols.}
  \label{fig:separate_p}
\end{figure}

%%% Local Variables: 
%%% mode: latex
%%% TeX-master: t
%%% End: 
